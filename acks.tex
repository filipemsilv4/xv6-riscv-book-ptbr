\chapter*{Prefácio e agradecimentos}

Este é um texto preliminar destinado a uma aula sobre sistemas operacionais. Ele
explica os conceitos principais de sistemas operacionais por meio do estudo de um exemplo
de kernel, chamado xv6. O xv6 é modelado com base no Unix Version 6 (v6) de Dennis Ritchie e
Ken Thompson~\cite{unix}. O xv6 segue de forma geral a estrutura
e estilo do v6, mas é implementado em ANSI C~\cite{kernighan} para
um RISC-V multi-core~\cite{riscv}.

Este texto deve ser lido juntamente com o código-fonte do xv6, uma
abordagem inspirada pelo "Commentary on UNIX 6th Edition" de John Lions~\cite{lions};
o texto possui hiperlinks para o código-fonte em
\url{https://github.com/mit-pdos/xv6-riscv}. Veja
\url{https://pdos.csail.mit.edu/6.1810} para mais informações e
recursos on-line sobre o v6 e o xv6, incluindo vários trabalhos práticos
usando o xv6.

Utilizamos este texto nas disciplinas 6.828 e 6.1810 sobre sistemas operacionais no MIT.
Agradecemos aos professores, assistentes de ensino e
estudantes dessas turmas que contribuíram direta ou indiretamente
para o xv6. Em particular, gostaríamos de agradecer a Adam Belay,
Austin Clements e Nickolai Zeldovich. Finalmente, gostaríamos de
agradecer às pessoas que nos enviaram e-mails reportando bugs no texto ou sugestões para
melhorias: Abutalib Aghayev, Sebastian Boehm, brandb97, Anton
Burtsev, Raphael Carvalho, Tej Chajed, Brendan Davidson, Rasit
Eskicioglu, Color Fuzzy, Wojciech Gac, Giuseppe, Tao Guo, Haibo Hao,
Naoki Hayama, Chris Henderson, Robert Hilderman, Eden Hochbaum,
Wolfgang Keller, Paweł Kraszewski, Henry Laih, Jin Li, Austin Liew,
lyazj@github.com, Pavan Maddamsetti, Jacek Masiulaniec, Michael
McConville, m3hm00d, miguelgvieira, Mark Morrissey, Muhammed Mourad,
Harry Pan, Harry Porter, Siyuan Qian, Zhefeng Qiao, Askar Safin,
Salman Shah, Huang Sha, Vikram Shenoy, Adeodato Simó, Ruslan
Savchenko, Pawel Szczurko, Warren Toomey, tyfkda, tzerbib, Vanush
Vaswani, Xi Wang e Zou Chang Wei, Sam Whitlock, Qiongsi Wu,
LucyShawYang, ykf1114@gmail.com e Meng Zhou.

Se você encontrar erros ou tiver sugestões para melhorias, por favor envie um e-mail para
Frans Kaashoek e Robert Morris (kaashoek, rtm@csail.mit.edu).