\documentclass[12pt]{book}
\usepackage[T1]{fontenc}
\usepackage{times}
\usepackage{listings}
\usepackage{graphicx}
\usepackage{xcolor}
\usepackage{url}
% \usepackage{showidx}
\usepackage{imakeidx}
\usepackage{booktabs}
\usepackage{url}
\usepackage{etoolbox}  % for showidx
\usepackage{fullpage}
\usepackage{soul}
\usepackage[utf8]{inputenc}

\usepackage{hyperref}  % should be last

% One space after periods
\frenchspacing

\hypersetup{pdfauthor={Russ Cox, Frans Kaashoek, Robert Morris},
            pdftitle={xv6: um sistema operacional didático, simples e semelhante ao Unix},}
  
\lstset{basicstyle=\small\ttfamily}
\lstset{morecomment=[is]{[[[}{]]]}}
\lstset{escapeinside={(*@}{@*)}}
\lstset{xleftmargin=5.0ex}

\newcommand{\github}{https://github.com/filipemsilv4/xv6-riscv-book-ptbr}

\newcommand{\fileref}[1]{\href{\github/#1}{\small{(#1)}}}
\newcommand{\lineref}[2]{\href{\github/#1\#L#2}{\small{(#1:#2)}}}
\newcommand{\linerefs}[3]{\href{\github/#1\#L#2-L#3}{\small(#1:#2-#3)}}

\newcommand{\indextext}[1]{\textit{#1}\index{#1}}
\newcommand{\indextextx}[1]{{#1}\index{#1}}
\newcommand{\indexcode}[1]{\lstinline{#1}\index{#1@\lstinline{#1}}}

%% editing markup
\newcommand{\insertnote}[3]{\noindent\textcolor{#1}{\textbf{#2:} #3}}
\newcommand{\note}[1]{\insertnote{blue}{NOTA}{#1}}
\newcommand{\rtm}[1]{\insertnote{red}{RTM}{#1}}
\newcommand{\mfk}[1]{\insertnote{red}{MFK}{#1}}
%% for publishing book without notes
%\renewcommand{\insertnote}[3]{}

\title{\textbf{xv6: um sistema operacional didático, simples e semelhante ao Unix}}
\author{Russ Cox \and Frans Kaashoek \and Robert Morris}

\makeindex

\begin{document}

\maketitle

\tableofcontents

\chapter*{Prefácio e agradecimentos}

Este é um texto preliminar destinado a uma aula sobre sistemas operacionais. Ele
explica os conceitos principais de sistemas operacionais por meio do estudo de um exemplo
de kernel, chamado xv6. O xv6 é modelado com base no Unix Version 6 (v6) de Dennis Ritchie e
Ken Thompson~\cite{unix}. O xv6 segue de forma geral a estrutura
e estilo do v6, mas é implementado em ANSI C~\cite{kernighan} para
um RISC-V multi-core~\cite{riscv}.

Este texto deve ser lido juntamente com o código-fonte do xv6, uma
abordagem inspirada pelo "Commentary on UNIX 6th Edition" de John Lions~\cite{lions};
o texto possui hiperlinks para o código-fonte em
\url{https://github.com/mit-pdos/xv6-riscv}. Veja
\url{https://pdos.csail.mit.edu/6.1810} para mais informações e
recursos on-line sobre o v6 e o xv6, incluindo vários trabalhos práticos
usando o xv6.

Utilizamos este texto nas disciplinas 6.828 e 6.1810 sobre sistemas operacionais no MIT.
Agradecemos aos professores, assistentes de ensino e
estudantes dessas turmas que contribuíram direta ou indiretamente
para o xv6. Em particular, gostaríamos de agradecer a Adam Belay,
Austin Clements e Nickolai Zeldovich. Finalmente, gostaríamos de
agradecer às pessoas que nos enviaram e-mails reportando bugs no texto ou sugestões para
melhorias: Abutalib Aghayev, Sebastian Boehm, brandb97, Anton
Burtsev, Raphael Carvalho, Tej Chajed, Brendan Davidson, Rasit
Eskicioglu, Color Fuzzy, Wojciech Gac, Giuseppe, Tao Guo, Haibo Hao,
Naoki Hayama, Chris Henderson, Robert Hilderman, Eden Hochbaum,
Wolfgang Keller, Paweł Kraszewski, Henry Laih, Jin Li, Austin Liew,
lyazj@github.com, Pavan Maddamsetti, Jacek Masiulaniec, Michael
McConville, m3hm00d, miguelgvieira, Mark Morrissey, Muhammed Mourad,
Harry Pan, Harry Porter, Siyuan Qian, Zhefeng Qiao, Askar Safin,
Salman Shah, Huang Sha, Vikram Shenoy, Adeodato Simó, Ruslan
Savchenko, Pawel Szczurko, Warren Toomey, tyfkda, tzerbib, Vanush
Vaswani, Xi Wang e Zou Chang Wei, Sam Whitlock, Qiongsi Wu,
LucyShawYang, ykf1114@gmail.com e Meng Zhou.

Se você encontrar erros ou tiver sugestões para melhorias, por favor envie um e-mail para
Frans Kaashoek e Robert Morris (kaashoek, rtm@csail.mit.edu).
\chapter{Interfaces do sistema operacional}
\label{CH:UNIX}

A tarefa de um sistema operacional é compartilhar um computador entre
múltiplos programas e fornecer um conjunto de serviços mais úteis
do que o hardware sozinho suporta.
Um sistema operacional gerencia e abstrai
o hardware de baixo nível, de modo que, por exemplo,
um processador de texto não precise se preocupar com qual tipo
de hardware de disco está sendo usado.
Um sistema operacional compartilha o hardware entre vários programas para
que eles sejam executados (ou pareçam ser executados) ao mesmo tempo.
Finalmente, os sistemas operacionais fornecem maneiras controladas para os programas
interagirem, para que possam compartilhar dados ou trabalhar juntos.

Um sistema operacional fornece serviços aos programas do usuário por meio de uma interface.
Projetar uma boa interface \index{interface design} acaba sendo
difícil. Por um lado, gostaríamos que a interface fosse
simples e restrita, porque isso torna mais fácil implementar corretamente.  Por outro lado,
podemos ser tentados a oferecer muitos recursos sofisticados aos aplicativos.
O truque para
resolver essa tensão é projetar interfaces que dependem de poucos
mecanismos que podem ser combinados para fornecer muita generalidade.

Este livro usa um único sistema operacional como um exemplo concreto para
ilustrar os conceitos de sistema operacional.  Esse sistema operacional,
xv6, fornece as interfaces básicas introduzidas pelo sistema operacional Unix de Ken Thompson e
Dennis Ritchie~\cite{unix}, bem como imita o
projeto interno do Unix.  O Unix fornece uma
interface restrita cujos mecanismos se combinam bem, oferecendo um grau surpreendente de
generalidade.  Esta interface tem sido tão bem-sucedida que
sistemas operacionais modernos—BSD, Linux, macOS, Solaris e até, em
menor grau, Microsoft Windows—têm interfaces semelhantes ao Unix.
Entender o xv6 é um bom começo para entender qualquer um desses
sistemas e muitos outros.

Como
mostra a Figura~\ref{fig:os},
o xv6 assume a forma tradicional de um
\indextext{núcleo (kernel)},
um programa especial que fornece
serviços para programas em execução.
Cada programa em execução, chamado de
\indextext{processo (process)},
tem memória contendo instruções, dados e uma pilha. As
instruções implementam a
computação do programa.  Os dados são as variáveis sobre as quais
a computação atua. A pilha organiza as chamadas de função do programa.
Um determinado computador normalmente tem muitos processos, mas apenas um único
núcleo.

Quando um
processo precisa invocar um serviço do núcleo, ele invoca
uma \indextext{chamada de sistema (system call)},
uma das chamadas
na interface do sistema operacional.
A chamada de sistema entra no núcleo;
o núcleo executa o serviço e retorna.
Assim, um processo alterna entre a execução no
\indextext{espaço do usuário (user space)}
e
\indextext{espaço do núcleo (kernel space)}.

Conforme descrito em detalhes nos capítulos subsequentes, o núcleo usa os mecanismos de proteção de hardware fornecidos por uma
CPU\footnote{
Este texto geralmente se refere ao elemento de hardware que executa um
cálculo com o termo \indextext{CPU}, uma sigla para unidade central de
processamento.  Outra documentação (por exemplo, a especificação RISC-V)
também usa as palavras processador, núcleo e hart em vez de CPU.
}
para
garantir que cada processo em execução no espaço do usuário possa acessar apenas
sua própria memória.
O núcleo é executado com os privilégios de hardware necessários para
implementar essas proteções; programas de usuário são executados sem
esses privilégios.
Quando um programa de usuário invoca uma chamada de sistema, o hardware
eleva o nível de privilégio e começa a executar uma
função pré-definida no núcleo.

\begin{figure}[t]
\center
\includegraphics[scale=0.5]{fig/os.pdf}
\caption{Um núcleo e dois processos de usuário.}
\label{fig:os}
\end{figure}

O conjunto de chamadas de sistema que um núcleo fornece
é a interface que os programas do usuário veem.
O núcleo xv6 fornece um subconjunto dos serviços e chamadas de sistema
que os núcleos Unix tradicionalmente oferecem. A
Figura~\ref{fig:api}
lista todas as chamadas de sistema do xv6.

O restante deste capítulo descreve os serviços do xv6 --- processos, memória,
descritores de arquivos, pipes e um sistema de arquivos --- e os ilustra com
trechos de código e discussões sobre como o \indextext{shell},
a interface de linha de comando do Unix, os usa.
O uso de chamadas de sistema pelo shell ilustra como elas foram cuidadosamente
projetadas.

O shell é um programa comum que lê comandos do usuário
e os executa.
O fato de o shell ser um programa de usuário, e não parte do núcleo,
ilustra o poder da interface de chamada de sistema: não há nada
de especial no shell.
Também significa que o shell é fácil de substituir; como resultado,
os sistemas Unix modernos têm uma variedade de
shells para escolher, cada um com sua própria interface de usuário
e recursos de script.
O shell xv6 é uma implementação simples da essência do
Bourne shell do Unix.  Sua implementação pode ser encontrada em
\lineref{user/sh.c:1}.
%%
%% 	Processos e memória
%%
\section{Processos e memória}

Um processo xv6 consiste em memória de espaço do usuário (instruções, dados e pilha)
e estado por processo privado para o núcleo.
Xv6 \indextext{divide o tempo (time-shares)} entre os processos: ele alterna de forma transparente as CPUs disponíveis
entre o conjunto de processos aguardando execução.
Quando um processo não está em execução, o xv6 salva os registradores da CPU do processo,
restaurando-os na próxima vez que executa o processo.
O núcleo associa um identificador de processo, ou
\indexcode{PID},
com cada processo.

\begin{figure}[H]
\center
\small
\begin{tabular}{p{4cm} p{10cm}} % Defina a largura de cada coluna conforme necessário
{\bf Chamada de Sistema (System call)} & {\bf Descrição} \\
\midrule
int fork() & Cria um processo, retorna o PID do filho. \\
int exit(int status) & Encerra o processo atual; status reportado para wait(). Sem retorno. \\
int wait(int *status) & Aguarda um filho encerrar; status de saída colocado em *status; retorna o PID do filho. \\
int kill(int pid) & Encerra o processo PID. Retorna 0, ou -1 para erro. \\
int getpid() & Retorna o PID do processo atual. \\
int sleep(int n) & Pausa por n pulsos de clock. \\
int exec(char *file, char *argv[]) & Carrega um arquivo e o executa com argumentos; retorna apenas se houver erro. \\
char *sbrk(int n) & Aumenta a memória do processo em n bytes zero. Retorna o início da nova memória. \\
int open(char *file, int flags) & Abre um arquivo; flags indicam leitura/gravação; retorna um fd (descritor de arquivo). \\
int write(int fd, char *buf, int n) & Grava n bytes de buf para o descritor de arquivo fd; retorna n. \\
int read(int fd, char *buf, int n) & Lê n bytes em buf; retorna o número lido; ou 0 se for o final do arquivo. \\
int close(int fd) & Libera o arquivo aberto fd. \\
int dup(int fd) & Retorna um novo descritor de arquivo referindo-se ao mesmo arquivo que fd. \\
int pipe(int p[]) & Cria um pipe, coloca os descritores de arquivo de leitura/gravação em p[0] e p[1]. \\
int chdir(char *dir) & Altera o diretório atual. \\
int mkdir(char *dir) & Cria um novo diretório. \\
int mknod(char *file, int, int) & Cria um arquivo de dispositivo. \\
int fstat(int fd, struct stat *st) & Coloca informações sobre um arquivo aberto em *st. \\
int link(char *file1, char *file2) & Cria outro nome (file2) para o arquivo file1. \\
int unlink(char *file) & Remove um arquivo. \\
\end{tabular}
\normalsize
\caption{Chamadas de sistema do Xv6. Se não declarado de outra forma, essas chamadas retornam
0 para nenhum erro e -1 se houver um erro.}
\label{fig:api}
\end{figure}

Um processo pode criar um novo processo usando a
chamada de sistema \indexcode{fork}.
\lstinline{fork}
fornece ao novo processo uma cópia exata da memória do processo
chamador: ele copia as instruções, os dados e
a pilha do processo chamador para a memória do novo processo.
\lstinline{fork}
retorna tanto no processo original quanto no novo.
No processo original, \lstinline{fork} retorna o PID do novo processo.
No novo processo, \lstinline{fork} retorna zero.
Os processos original e novo são frequentemente chamados de
\indextext{pai (parent)}
e
\indextext{filho (child)}.

Por exemplo, considere o seguinte fragmento de programa escrito na linguagem de programação C
~\cite{kernighan}:
\begin{lstlisting}[]
int pid = fork();
if(pid > 0){
  printf("parent: child=%d\n", pid);
  pid = wait((int *) 0);
  printf("child %d is done\n", pid);
} else if(pid == 0){
  printf("child: exiting\n");
  exit(0);
} else {
  printf("fork error\n");
}
\end{lstlisting}
A
chamada de sistema \indexcode{exit}
faz com que o processo que a chamou pare de executar e
libere recursos como memória e arquivos abertos.
\lstinline{exit} recebe um argumento de status inteiro,
convencionalmente 0 para indicar sucesso e 1 para indicar falha.
A
chamada de sistema \indexcode{wait}
retorna o PID de um filho encerrado (ou finalizado) do
processo atual e copia o status de saída do filho para o endereço
passado para \lstinline{wait}; se nenhum dos filhos do chamador
tiver encerrado,
\indexcode{wait}
espera que um o faça.
Se o chamador não tiver filhos, \lstinline{wait} retorna imediatamente -1.
Se o pai não se importar com o status de saída de um filho, ele pode
passar um endereço 0 para
\lstinline{wait}.

No exemplo, as linhas de saída
\begin{lstlisting}[]
parent: child=1234
child: exiting
\end{lstlisting}
podem sair em qualquer ordem (ou até mesmo intercaladas), dependendo se o
pai ou o filho chega à sua
chamada \indexcode{printf}
primeiro.
Após o filho encerrar, o \indexcode{wait} do pai
retorna, fazendo com que o pai imprima
\begin{lstlisting}[]
parent: child 1234 is done
\end{lstlisting}
Embora o filho tenha o mesmo conteúdo de memória que o pai inicialmente, o
pai e o filho estão executando com memória separada e registradores separados:
alterar uma variável em um não afeta o outro. Por exemplo, quando o
valor de retorno de
\lstinline{wait}
é armazenado em
\lstinline{pid}
no processo pai,
ele não altera a variável
\lstinline{pid}
no filho.  O valor de
\lstinline{pid}
no filho ainda será zero.

A
chamada de sistema \indexcode{exec}
substitui a memória do processo que a chamou por uma nova imagem de memória
carregada de um arquivo armazenado no sistema de arquivos.
O arquivo deve ter um formato específico, que especifica qual parte do
arquivo contém instruções, qual parte são dados, em qual instrução
iniciar, etc. O Xv6
usa o formato ELF, que o Capítulo~\ref{CH:MEM} discute em
mais detalhes.
Normalmente, o arquivo é o resultado da compilação do código-fonte de um programa.
Quando
\indexcode{exec}
é bem-sucedido, ele não retorna ao programa de chamada;
em vez disso, as instruções carregadas do arquivo iniciam
a execução no ponto de entrada declarado no cabeçalho ELF.
\lstinline{exec}
recebe dois argumentos: o nome do arquivo que contém o
executável e uma array de strings de argumentos.
Por exemplo:
\begin{lstlisting}[]
char *argv[3];

argv[0] = "echo";
argv[1] = "hello";
argv[2] = 0;
exec("/bin/echo", argv);
printf("exec error\n");
\end{lstlisting}
Este fragmento substitui o programa de chamada por uma instância
do programa
\lstinline{/bin/echo}
executando com a lista de argumentos
\lstinline{echo}
\lstinline{hello}.
A maioria dos programas ignora o primeiro elemento da array de argumentos, que é
convencionalmente o nome do programa.

A shell do xv6 utiliza as chamadas mencionadas acima para executar programas para
os usuários. A estrutura principal do shell é simples; veja
\lstinline{main}
\lineref{user/sh.c:/main/}.
O loop principal lê uma linha de entrada do usuário com
\indexcode{getcmd}.
Então, ele chama
\lstinline{fork},
que cria uma cópia do processo do shell. O
pai chama
\lstinline{wait},
enquanto o filho executa o comando.  Por exemplo, se o usuário
tivesse digitado
``\lstinline{echo hello}''
para o shell,
\lstinline{runcmd}
teria sido chamado com
``\lstinline{echo hello}''
como argumento.
\lstinline{runcmd}
\lineref{user/sh.c:/runcmd/}
executa o comando real. Para
``\lstinline{echo hello}'',
ele chamaria
\lstinline{exec}
\lineref{user/sh.c:/exec.ecmd/}.
Se
\lstinline{exec}
for bem-sucedido, o filho executará as instruções de
\lstinline{echo}
em vez de
\lstinline{runcmd}.  Em algum ponto
\lstinline{echo}
chamará
\lstinline{exit},
o que fará com que o pai retorne de
\lstinline{wait}
em
\lstinline{main}
\lineref{user/sh.c:/main/}.


Você pode se perguntar por que
\indexcode{fork}
e
\indexcode{exec}
não são combinados em uma única chamada; veremos mais tarde que
o shell explora a separação em sua implementação de
redirecionamento de E/S.
Para evitar o desperdício de
criar um processo duplicado e substituí-lo imediatamente (com \lstinline{exec}),
os núcleos do sistema operacional otimizam a implementação de
\lstinline{fork}
para este caso de uso usando técnicas de memória virtual como
cópia na gravação (copy-on-write) (consulte a Seção~\ref{sec:pagefaults}).

Xv6 aloca a maior parte da memória do espaço do usuário
implicitamente:
\indexcode{fork}
aloca a memória necessária para a cópia do filho da
memória do pai, e
\indexcode{exec}
aloca memória suficiente para armazenar o arquivo executável.
Um processo que precisa de mais memória em tempo de execução (talvez para
\indexcode{malloc})
pode chamar
\lstinline{sbrk(n)}
para aumentar sua memória de dados em
\lstinline{n} bytes zero;
\indexcode{sbrk}
retorna a localização da nova memória.

%% 
%% 	I/O and File descriptors
%% 
\section{I/O and File descriptors}

A 
\indextext{file descriptor} 
is a small integer representing a kernel-managed object
that a process may read from or write to.
A process may obtain a file descriptor by opening a file, directory,
or device, or by creating a pipe, or by duplicating an existing
descriptor.
For simplicity we'll often refer to the object a file descriptor
refers to as a ``file'';
the file descriptor interface abstracts away the differences between
files, pipes, and devices, making them all look like streams of bytes.
We'll refer to input and output as \indextext{I/O}.

Internally, the xv6 kernel uses the file descriptor
as an index into a per-process table,
so that every process has a private space of file descriptors
starting at zero.
By convention, a process reads from file descriptor 0 (standard input),
writes output to file descriptor 1 (standard output), and
writes error messages to file descriptor 2 (standard error).
As we will see, the shell exploits the convention to implement I/O redirection
and pipelines. The shell ensures that it always has three file descriptors
open
\lineref{user/sh.c:/open..console/},
which are by default file descriptors for the console.

The
\lstinline{read}
and
\lstinline{write}
system calls read bytes from and write bytes to
open files named by file descriptors.
The call
\lstinline{read(fd},
\lstinline{buf},
\lstinline{n)}
reads at most
\lstinline{n}
bytes from the file descriptor
\lstinline{fd},
copies them into
\lstinline{buf},
and returns the number of bytes read.
Each file descriptor that refers to a file
has an offset associated with it.
\lstinline{read}
reads data from the current file offset and then advances
that offset by the number of bytes read:
a subsequent
\lstinline{read}
will return the bytes following the ones returned by the first
\lstinline{read}.
When there are no more bytes to read,
\lstinline{read}
returns zero to indicate the end of the file.

The call
\lstinline{write(fd},
\lstinline{buf},
\lstinline{n)}
writes
\lstinline{n}
bytes from
\lstinline{buf}
to the file descriptor
\lstinline{fd}
and returns the number of bytes written.
Fewer than
\lstinline{n}
bytes are written only when an error occurs.
Like
\lstinline{read},
\lstinline{write}
writes data at the current file offset and then advances
that offset by the number of bytes written:
each
\lstinline{write}
picks up where the previous one left off.

The following program fragment (which forms the essence of the program
\lstinline{cat})
copies data from its standard input
to its standard output.  If an error occurs, it writes a message
to the standard error.
\begin{lstlisting}[]
char buf[512];
int n;

for(;;){
  n = read(0, buf, sizeof buf);
  if(n == 0)
    break;
  if(n < 0){
    fprintf(2, "read error\n");
    exit(1);
  }
  if(write(1, buf, n) != n){
    fprintf(2, "write error\n");
    exit(1);
  }
}
\end{lstlisting}
The important thing to note in the code fragment is that
\lstinline{cat}
doesn't know whether it is reading from a file, console, or a pipe.
Similarly 
\lstinline{cat}
doesn't know whether it is printing to a console, a file, or whatever.
The use of file descriptors and the convention that file descriptor 0
is input and file descriptor 1 is output allows a simple
implementation
of 
\lstinline{cat}.

The
\lstinline{close}
system call
releases a file descriptor, making it free for reuse by a future
\lstinline{open},
\lstinline{pipe},
or
\lstinline{dup}
system call (see below).
A newly allocated file descriptor 
is always the lowest-numbered unused
descriptor of the current process.

File descriptors and
\indexcode{fork}
interact to make I/O redirection easy to implement.
\lstinline{fork}
copies the parent's file descriptor table along with its memory,
so that the child starts with exactly the same open files as the parent.
The system call
\indexcode{exec}
replaces the calling process's memory but preserves its file table.
This behavior allows the shell to
implement \indextext{I/O redirection} by forking,
re-opening chosen file descriptors in the child,
and then calling \lstinline{exec} to run the new program.
Here is a simplified version of the code a shell runs for the
command
\lstinline{cat}
\lstinline{<}
\lstinline{input.txt}:
\begin{lstlisting}[]
char *argv[2];

argv[0] = "cat";
argv[1] = 0;
if(fork() == 0) {
  close(0);
  open("input.txt", O_RDONLY);
  exec("cat", argv);
}
\end{lstlisting}
After the child closes file descriptor 0,
\lstinline{open}
is guaranteed to use that file descriptor
for the newly opened
\lstinline{input.txt}:
0 will be the smallest available file descriptor.
\lstinline{cat}
then executes with file descriptor 0 (standard input) referring to
\lstinline{input.txt}.
The parent process's file descriptors are not changed by this
sequence, since it modifies only the child's descriptors.

The code for I/O redirection in the xv6 shell works in exactly this way
\lineref{user/sh.c:/case.REDIR/}.
Recall that at this point in the code the shell has already forked the
child shell and that 
\lstinline{runcmd} 
will call
\lstinline{exec}
to load the new program.

The second argument to \lstinline{open} consists of a set of
flags, expressed as bits, that control what \lstinline{open}
does. The possible values are defined in the file control (fcntl) header
\linerefs{kernel/fcntl.h:/RDONLY/,/TRUNC/}:
\lstinline{O_RDONLY},
\lstinline{O_WRONLY},
\lstinline{O_RDWR},
\lstinline{O_CREATE}, and
\lstinline{O_TRUNC},
which instruct \lstinline{open} to
open the file for reading,
or for writing,
or for both reading and writing,
to create the file if it doesn't exist,
and to truncate the file to zero length.

Now it should be clear why it is helpful that
\lstinline{fork}
and 
\lstinline{exec} 
are separate calls: between the two, the shell has a chance
to redirect the child's I/O without disturbing the I/O setup of the main shell.
One could instead imagine a hypothetical combined
\lstinline{forkexec} system call,
but the options for doing I/O redirection with such a call
seem awkward.
The shell could modify its own I/O
setup before calling \lstinline{forkexec} (and then
un-do those modifications); or
\lstinline{forkexec} could take instructions for I/O
redirection as arguments;
or (least attractively) every program like \lstinline{cat} could
be taught to do its own I/O redirection.

Although
\lstinline{fork}
copies the file descriptor table, each underlying file offset is shared
between parent and child.
Consider this example:
\begin{lstlisting}[]
if(fork() == 0) {
  write(1, "hello ", 6);
  exit(0);
} else {
  wait(0);
  write(1, "world\n", 6);
}
\end{lstlisting}
At the end of this fragment, the file attached to file descriptor 1
will contain the data
\lstinline{hello}
\lstinline{world}.
The
\lstinline{write}
in the parent
(which, thanks to
\lstinline{wait},
runs only after the child is done)
picks up where the child's
\lstinline{write}
left off.
This behavior helps produce sequential output from sequences
of shell commands, like
\lstinline{(echo}
\lstinline{hello};
\lstinline{echo}
\lstinline{world)}
\lstinline{>output.txt}.

The
\lstinline{dup}
system call duplicates an existing file descriptor,
returning a new one that refers to the same underlying I/O object.
Both file descriptors share an offset, just as the file descriptors
duplicated by
\lstinline{fork}
do.
This is another way to write
\lstinline{hello}
\lstinline{world}
into a file:
\begin{lstlisting}[]
fd = dup(1);
write(1, "hello ", 6);
write(fd, "world\n", 6);
\end{lstlisting}

Two file descriptors share an offset if they were derived from
the same original file descriptor by a sequence of
\lstinline{fork}
and
\lstinline{dup}
calls.
Otherwise file descriptors do not share offsets, even if they
resulted from 
\lstinline{open}
calls for the same file.  
\lstinline{dup} 
allows shells to implement commands like this:
\lstinline{ls}
\lstinline{existing-file}
\lstinline{non-existing-file}
\lstinline{>}
\lstinline{tmp1}
\lstinline{2>&1}.
The
\lstinline{2>&1}
tells the shell to give the command a file descriptor 2 that
is a duplicate of descriptor 1.
Both the name of the existing file and the error message for the
non-existing file will show up in the file
\lstinline{tmp1}.
The xv6 shell doesn't support I/O redirection for the error file
descriptor, but now you know how to implement it.

File descriptors are a powerful abstraction,
because they hide the details of what they are connected to:
a process writing to file descriptor 1 may be writing to a
file, to a device like the console, or to a pipe.
%% 
%% 	Pipes
%% 
\section{Pipes}

A 
\indextext{pipe} 
is a small kernel buffer exposed to processes as a pair of
file descriptors, one for reading and one for writing.
Writing data to one end of the pipe
makes that data available for reading from the other end of the pipe.
Pipes provide a way for processes to communicate.

The following example code runs the program
\lstinline{wc}
with standard input connected to
the read end of a pipe.
\begin{lstlisting}[]
int p[2];
char *argv[2];

argv[0] = "wc";
argv[1] = 0;

pipe(p);
if(fork() == 0) {
  close(0);
  dup(p[0]);
  close(p[0]);
  close(p[1]);
  exec("/bin/wc", argv);
} else {
  close(p[0]);
  write(p[1], "hello world\n", 12);
  close(p[1]);
}
\end{lstlisting}
The program calls
\lstinline{pipe},
which creates a new pipe and records the read and write
file descriptors in the array
\lstinline{p}.
After
\lstinline{fork},
both parent and child have file descriptors referring to the pipe.
The child calls \lstinline{close} and \lstinline{dup} to make file descriptor
zero refer to the read end of the pipe,
closes the file descriptors in
\lstinline{p},
and calls \lstinline{exec} to run
\lstinline{wc}.
When 
\lstinline{wc}
reads from its standard input, it reads from the pipe.
The parent closes the read side of the pipe,
writes to the pipe,
and then closes the write side.

If no data is available, a
\lstinline{read}
on a pipe waits for either data to be written or for all
file descriptors referring to the write end to be closed;
in the latter case,
\lstinline{read}
will return 0, just as if the end of a data file had been reached.
The fact that
\lstinline{read}
blocks until it is impossible for new data to arrive
is one reason that it's important for the child to
close the write end of the pipe
before executing
\lstinline{wc}
above: if one of
\lstinline{wc} 's
file descriptors referred to the write end of the pipe,
\lstinline{wc}
would never see end-of-file.

The xv6 shell implements pipelines such as
\lstinline{grep fork sh.c | wc -l}
in a manner similar to the above code
\lineref{user/sh.c:/case.PIPE/}.
The child process creates a pipe to connect the left end of the pipeline
with the right end. Then it calls
\lstinline{fork}
and
\lstinline{runcmd}
for the left end of the pipeline
and 
\lstinline{fork}
and
\lstinline{runcmd}
for the right end, and waits for both to finish.
The right end of the pipeline may be a command that itself includes a
pipe (e.g.,
\lstinline{a}
\lstinline{|}
\lstinline{b}
\lstinline{|}
\lstinline{c)}, 
which itself forks two new child processes (one for
\lstinline{b}
and one for
\lstinline{c}).
Thus, the shell may
create a tree of processes.  The leaves of this tree are commands and
the interior nodes are processes that wait until the left and right
children complete.

% In principle, one could have the interior nodes run the left end of a
% pipeline, but doing so correctly would complicate the
% implementation. Consider making just the following modification:
% change \lstinline{sh.c} to not fork for \lstinline{p->left} and run
% \lstinline{runcmd(p->left)} in the interior process. Then, for
% example, \lstinline{echo hi | wc} won't produce output, because when
% \lstinline{echo hi} exits in \lstinline{runcmd}, the interior process
% exits and never calls fork to run the right end of the pipe.  This
% incorrect behavior could be fixed by not calling \lstinline{exit} in
% \lstinline{runcmd} for interior processes, but this fix complicates
% the code: now \lstinline{runcmd} needs to know if it's in an interior
% process or not.  Complications also arise when not forking for
% \lstinline{runcmd(p->right)}.  For example, with just that
% modification, \lstinline{sleep 10 | echo hi} will immediately print
% ``hi' and a new prompt, instead of after 10 seconds; this happens because \lstinline{echo} runs
% immediately and exits, not waiting for \lstinline{sleep} to finish.
% Since the goal of the \lstinline{sh.c} is to be as simple as possible,
% it doesn't try to avoid creating interior processes.

Pipes may seem no more powerful than temporary files:
the pipeline
\begin{lstlisting}[]
echo hello world | wc
\end{lstlisting}
could be implemented without pipes as
\begin{lstlisting}[]
echo hello world >/tmp/xyz; wc </tmp/xyz
\end{lstlisting}
Pipes have at least three advantages over temporary files
in this situation.
First, pipes automatically clean themselves up;
with the file redirection, a shell would have to
be careful to remove
\lstinline{/tmp/xyz}
when done.
Second, pipes can pass arbitrarily long streams of
data, while file redirection requires enough free space
on disk to store all the data.
Third, pipes allow for parallel execution of pipeline stages,
while the file approach requires the first program to finish
before the second starts.
% Fourth, if you are implementing inter-process communication,
% pipes' blocking reads and writes are more efficient
% than the non-blocking semantics of files.
%% 
%% 	File system
%% 
\section{File system}

The xv6 file system provides data files,
which contain uninterpreted byte arrays,
and directories, which
contain named references to data files and other directories.
The directories form a tree, starting
at a special directory called the 
\indextext{root}.
A 
\indextext{path} 
like
\lstinline{/a/b/c}
refers to the file or directory named
\lstinline{c}
inside the directory named
\lstinline{b}
inside the directory named
\lstinline{a}
in the root directory
\lstinline{/}.
Paths that don't begin with
\lstinline{/}
are evaluated relative to the calling process's
\indextext{current directory},
which can be changed with the
\lstinline{chdir}
system call.
Both these code fragments open the same file
(assuming all the directories involved exist):
\begin{lstlisting}[]
chdir("/a");
chdir("b");
open("c", O_RDONLY);

open("/a/b/c", O_RDONLY);
\end{lstlisting}
The first fragment changes the process's current directory to
\lstinline{/a/b};
the second neither refers to nor changes the process's current directory.

There are system calls to create new files and directories:
\lstinline{mkdir}
creates a new directory,
\lstinline{open}
with the
\lstinline{O_CREATE}
flag creates a new data file,
and
\lstinline{mknod}
creates a new device file.
This example illustrates all three:
\begin{lstlisting}[]
mkdir("/dir");
fd = open("/dir/file", O_CREATE|O_WRONLY);
close(fd);
mknod("/console", 1, 1);
\end{lstlisting}
\lstinline{mknod}
creates a special file that refers to a device.
Associated with a device file are
the major and minor device numbers
(the two arguments to 
\lstinline{mknod}),
which uniquely identify a kernel device.
When a process later opens a device file, the kernel
diverts
\lstinline{read}
and
\lstinline{write}
system calls to the kernel device implementation
instead of passing them to the file system.

A file's name is distinct from the file itself;
the same underlying file, called an 
\indextext{inode}, 
can have multiple names,
called 
\indextext{links}.
Each link consists of an entry in a directory;
the entry contains a file name and a reference
to an inode.
An inode holds
\indextext{metadata}
about a file, including
its type (file or directory or device),
its length,
the location of the file's content on disk,
and the number of links to a file.

The
\lstinline{fstat}
system call
retrieves information from the inode that a
file descriptor refers to.
It fills in a
\lstinline{struct}
\lstinline{stat},
defined in
\lstinline{stat.h} \fileref{kernel/stat.h}
as:
\begin{lstlisting}[]
#define T_DIR     1   // Directory
#define T_FILE    2   // File
#define T_DEVICE  3   // Device

struct stat {
  int dev;     // File system's disk device
  uint ino;    // Inode number
  short type;  // Type of file
  short nlink; // Number of links to file
  uint64 size; // Size of file in bytes
};
\end{lstlisting}

The
\lstinline{link}
system call creates another file system name 
referring to the same inode as an existing file.
This fragment creates a new file named both
\lstinline{a}
and
\lstinline{b}.
\begin{lstlisting}[]
open("a", O_CREATE|O_WRONLY);
link("a", "b");
\end{lstlisting}
Reading from or writing to
\lstinline{a}
is the same as reading from or writing to
\lstinline{b}.
Each inode is identified by a unique
\textit{inode}
\textit{number}.
After the code sequence above, it is possible
to determine that
\lstinline{a}
and
\lstinline{b}
refer to the same underlying contents by inspecting the
result of 
\lstinline{fstat}:
both will return the same inode number 
(\lstinline{ino}),
and the
\lstinline{nlink}
count will be set to 2.

The
\lstinline{unlink}
system call removes a name from the file system.
The file's inode and the disk space holding its content
are only freed when the file's link count is zero and
no file descriptors refer to it.
Thus adding
\begin{lstlisting}[]
unlink("a");
\end{lstlisting}
to the last code sequence leaves the inode
and file content accessible as
\lstinline{b}.
Furthermore,
\begin{lstlisting}[]
fd = open("/tmp/xyz", O_CREATE|O_RDWR);
unlink("/tmp/xyz");
\end{lstlisting}
is an idiomatic way to create a temporary inode 
with no name
that will be cleaned up when the process closes 
\lstinline{fd}
or exits.

Unix provides
file utilities callable from the shell
as user-level programs, for example
\lstinline{mkdir},
\lstinline{ln},
and
\lstinline{rm}.
This design allows anyone to extend the command-line interface
by adding new user-level programs.  In hindsight this plan seems obvious,
but other systems designed at the time of Unix often built such commands into
the shell (and built the shell into the kernel).

One exception is
\lstinline{cd},
which is built into the shell
\lineref{user/sh.c:/if.buf.0..==..c./}.
\lstinline{cd}
must change the current working directory of the
shell itself.  If
\lstinline{cd}
were run as a regular command, then the shell would fork a child
process, the child process would run
\lstinline{cd},
and
\lstinline{cd}
would change the 
\textit{child} 's 
working directory.  The parent's (i.e.,
the shell's) working directory would not change.
%% 
%% 	Real world
%% 
\section{Real world}

Unix's combination of ``standard'' file
descriptors, pipes, and convenient shell syntax for
operations on them was a major advance in writing
general-purpose reusable programs.
The idea sparked a culture of ``software tools'' that was
responsible for much of Unix's power and popularity,
and the shell was the first so-called ``scripting language.''
The Unix system call interface persists today in systems like
BSD, Linux, and macOS.

The Unix system call interface has been standardized through the Portable
Operating System Interface (POSIX) standard.
Xv6 is
\textit{not}
POSIX compliant:  it is missing many system calls (including basic ones such as
\lstinline{lseek}),
and many of the system calls it does provide differ from the standard.
Our main goals for xv6 are
simplicity and clarity while providing a simple UNIX-like system-call interface.
Several people have extended xv6 with a few more system calls and a simple
C library in order to run basic Unix programs.  Modern kernels, however,
provide many more system calls, and many more kinds of kernel services, than
xv6.  For example, they support networking, windowing systems, user-level threads,
drivers for many devices, and so on.  Modern kernels evolve continuously and
rapidly, and offer many features beyond POSIX.

Unix unified access to multiple types of resources (files,
directories, and devices) with a single set of 
file-name and file-descriptor interfaces.
This idea can be extended to more kinds of resources;
a good example is Plan 9~\cite{Presotto91plan9},
which applied the ``resources are files''
concept to 
networks, graphics, and more.
However, most Unix-derived operating systems have
not followed this route.

The file system and file descriptors have been  powerful
abstractions.
Even so, there are other models for operating system interfaces.
Multics, a predecessor of Unix,
abstracted file storage in a way that made it look like memory,
producing a very different flavor of interface.
The complexity of the Multics design had a direct influence
on the designers of Unix, who aimed to build something simpler.
% XXX can we cut this, since its point is the same as the next paragraph?
% An operating system interface that went out of fashion
% decades ago but has recently returned is the idea of a virtual machine monitor.
% Such systems provide a superficially different interface from xv6,
% but the basic concepts are still the same:
% a virtual machine, like a process, consists of some memory and
% one or more register sets;
% the virtual machine has access to one large file called
% a virtual disk instead of a file system;
% virtual machines send messages to each other
% and the outside world using virtual network devices
% instead of pipes or files.

Xv6 does not provide a notion of users or of protecting
one user from another; in Unix terms, all xv6 processes
run as root.

This book examines how xv6 implements its Unix-like interface,
but the ideas and concepts apply to more than just Unix.
Any operating system must multiplex processes onto
the underlying hardware, isolate processes from each
other, and provide mechanisms for controlled
inter-process communication.
After studying xv6, you should be able to
look at other, more complex operating systems
and see the concepts underlying xv6 in those systems as well.

%% 
\section{Exercises}
%%

\begin{enumerate}

\item Write a program that uses UNIX system calls to
``ping-pong'' a byte between two processes over a pair
of pipes, one for each direction. Measure the
program's performance, in exchanges per second.

\end{enumerate}

\input{latex.out/first}
\input{latex.out/mem}
\input{latex.out/trap}
\input{latex.out/interrupt}
\input{latex.out/lock}
\input{latex.out/sched}
\input{latex.out/fs}
\input{latex.out/lock2}
\input{latex.out/sum}

{
% The following prevents latex from splitting a bibliography entry with a page
% break
\interlinepenalty=10000
% Since we're using natbib in numbers mode, we don't need plainnat,
% which exists to feed authors and years back in to natbib.  As a
% result, it complains about entries without years, which we don't
% care about.
%\bibliographystyle{plainnat}
\bibliographystyle{plain}
\bibliography{book}
}

\printindex

\end{document}
